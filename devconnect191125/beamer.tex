
\documentclass[aspectratio=1610]{beamer}

\usepackage[utf8]{inputenc}
\usepackage[T1]{fontenc}
\usepackage{amsmath,amssymb,amsfonts}
\usepackage{hyperref} % before url package otherwise there is a pb
\usepackage{url}

\setbeamertemplate{navigation symbols}{%
    \usebeamerfont{footline}%
    \usebeamercolor[fg]{footline}%
    \hspace{1em}%
    \insertframenumber/\inserttotalframenumber
}
\setbeamercolor{footline}{fg=black}


\usepackage{comment}

% fix bug in algorithm2e
\makeatletter

\usepackage{tikz}
\usetikzlibrary{calc, positioning, arrows.meta}
\usetikzlibrary{shapes.geometric} % for ellipse
\usetikzlibrary{arrows}
\usetikzlibrary{positioning}
\tikzset{
  state/.style={
    rectangle,
    rounded corners,
    draw=black, thick,
    minimum height=2em,
    inner sep=2pt,
    text centered,
  },
}
\usetikzlibrary{tikzmark}
\usepackage{pgfplots}

\usepackage{multirow}
\usepackage{multicol}

\newcommand{\NN}{\ensuremath{\mathbb N}}
\newcommand{\ZZ}{\ensuremath{\mathbb Z}}
\newcommand{\QQ}{\ensuremath{\mathbb{Q}}}
\newcommand{\RR}{\ensuremath{\mathbb{R}}}
\newcommand{\CC}{\ensuremath{\mathbb{C}}}
\newcommand{\K}{\ensuremath{\mathbb{K}}}
\newcommand{\PP}{\ensuremath{\mathbb{P}}} % for projective line
\newcommand{\FF}{\ensuremath{\mathbb F}}
\newcommand{\F}{\ensuremath{\mathbb F}}
\newcommand{\GG}{\ensuremath{\mathbb G}}
\newcommand{\G}{\ensuremath{\mathbb G}}
\newcommand{\GT}{\ensuremath{\mathbb G_{\text{T}}}}
\newcommand{\Fp}{\F_p}
\newcommand{\cO}{\ensuremath{\mathcal O}}
\newcommand{\cF}{\ensuremath{\mathcal F}}
\DeclareMathOperator{\Norm}{Norm}
\DeclareMathOperator{\ord}{ord}
\DeclareMathOperator{\disc}{Disc}
\DeclareMathOperator{\coeff}{Coeff}
\DeclareMathOperator{\cont}{cont}
\DeclareMathOperator{\reslt}{Res}
\DeclareMathOperator{\Reslt}{Res}
\DeclareMathOperator{\Res}{Res}
\DeclareMathOperator{\val}{val}
\DeclareMathOperator{\GF}{GF}
\DeclareMathOperator{\aut}{aut}
\DeclareMathOperator{\End}{End}
\DeclareMathOperator{\lc}{lc} % for leading coefficient
\DeclareMathOperator{\Id}{Id} % for Identity
\DeclareMathOperator{\Div}{div}
\newcommand{\BLS}{\mbox{BLS}}

\usepackage{fontawesome}

\date{November 18th, 2025 -- Buenos Aires, DevConnect}
\title{
  Hacky Schnorr for MPC and ZK with applications to Kohaku wallet}
\author{\textbf{Simon Masson}, Renaud Dubois\\
ZKNox\\
\includegraphics[height=4em]{zknox.png}\\
Privacy and Compliance Summit}
\begin{document}
\begin{frame}
  \maketitle
\end{frame}

\begin{frame}{ZKNOX team}
  \begin{minipage}{.65\linewidth}
   \begin{tabular}{rl}
      \multirow{4}{*}{\includegraphics[width=2cm]{NB.jpeg}} & \textbf{Nicolas Bacca}\\
                               & \small{$20^+$ years experience ($10^+$y web3)}\\
                               & Security and hardware specialist\\
                               & \small{Prev.~Ledger cofounder/CTO}\\
                               & \\
      \multirow{4}{*}{\includegraphics[width=2cm]{RD.jpeg}} & \textbf{Renaud Dubois}\\
                               & \small{$20^+$ years experience ($3^+$y web3)}\\
                               & Cryptographer\\
                               & \small{Prev.~Ledger, Thales}\\
                               & \\
      \multirow{4}{*}{\includegraphics[width=2cm]{SM.jpg}} & \textbf{Simon Masson}\\
                               & \small{$8^+$ years experience ($4^+$y web3)}\\
                               & Cryptographer\\
                               & \small{Prev.~Heliax, Thales}\\
  \end{tabular}
\end{minipage}\pause
\begin{minipage}{.33\linewidth}
   Expertise and innovation to every challenge on the whole security chain:
   \begin{itemize}
   \item user end\\ (secure enclaves, hardware wallets),
   \item back end\\ (TEE, HSMs),
   \item on-chain\\ (smart contracts).
   \end{itemize}

  \pause
 
  \vspace{1em}

  \href{https://zknox.eth.limo/}{\texttt{https://zknox.eth.limo/}}

  \vspace{1em}

  \href{https://github.com/zknoxhq/}{\texttt{https://github.com/zknoxhq/}}

\end{minipage}
\end{frame}


\begin{frame}{Summary}
 \tableofcontents
\end{frame}  

% \begin{frame}{Content of this talk}
%   \begin{enumerate}
%     \item Introduction: privacy in Ethereum
%     \item Zero-knowledge circuits in a nutshell
%     \item Elliptic curve in circuits
%     \item Hinted scalar multiplications
%     \item Conclusion and perspectives
%   \end{enumerate}
% \end{frame}
 \section{Signatures, MultiSigs and ThresholdSigs }
    %\frame{\sectionpage}
    \subsection{Basic Concepts}
%%%%%%%%%%%%%%%%%%%%%%%%%%%%%%%%%%%%%%%%%%%%%%%%    
    \begin{frame}{Signatures}
     
%      \begin{definition}[Wikipedia]
%      Identity is the qualities, beliefs, personality traits, appearance, and/or expressions that characterize a person or group.
%      \end{definition}
     
     \only<1>
     {
       A digital signature is a mathematical scheme for verifying the authenticity of digital messages or documents.
     	\begin{center}
        \includegraphics[width=8cm]{images/signature.jpg}
        \end{center}
    
     }
           \only<2>{
            \begin{definition}[(Classical) Digital Signature]
     A signature scheme is a tuple of function:
     \begin{itemize}
     \item $Setup$: returns domain parameters $E(F_p), G, H$
     \item $KeyGen(E(F_p), G, H, seed)$: returns $(pvk,pubk)=(x,Q)$
     \item $Sign(x,message)$: returns $Sig$ 
     \item $Verify(Sig,Q)$: returns {\tt true/false}
     \end{itemize}
     
     \end{definition}
     \href{https://hyperelliptic.org/EFD/g1p/auto-shortw.html}{{Formulaes}} for elliptic computations.
     
     \href{https://neuromancer.sk/std/network}{{Dictionnary of curves parameters}} 
     
           }
      \only<3>
     {
     \begin{exampleblock}{Properties}
       \begin{itemize}
     \item Unforgeable
     \item Non repudiation
     \item {\emph{ Not reusable}}
     \end{itemize}
     \end{exampleblock}
     
      Most commonly used signature scheme is ECDSA (Bitcoin, Ethereum, Passkeys)
     \begin{itemize}
     \item A painfull patent prevented Schnorr from being used, now expired
     \item Schnorr is used in Taproot, Ed25519, EDDSA POSEIDON (CIRCOMLIB, RAILGUN)
     
     \end{itemize} 
     }
    
    
\end{frame}



%%%%%%%%%%%%%%%%%%%%%%%%%%%%%%%%%%%%%%%%%%%%%%%%
 
\begin{frame}{Threshold-signatures}


 
  \only<1>
  {
   A $(k,n)$ threshold signature (TS-Sig) is a digital signature allowing a subset (threshold) of $k$ users from $n$ to {\it aggregate} a signature . 
 
   \begin{center}
        \includegraphics[width=12.2cm]{images/tss.png}
        \end{center}
  }
  \only<2>
  {
    \begin{definition}[Threshold Signatures]
     A multisig scheme is a tuple of function:
     \begin{itemize}
     \item $(Setup,  Verify, Sign)$
     \item $Distributed Keygen$,
     \item $KeyAgg(Q_1, \ldots Q_n)$ returns $X$
     \item $SignAgg(Sig_1, \ldots, Sig_n)$ returns $Sig$
     \end{itemize}
  \end{definition}
  }

   
\end{frame}
%%%%%%%%%%%%%%%%%%%%%%%%%%%%%%%%%%%%%%%%%%%%%%%%
 \subsection{Under the hood}

\begin{frame}{Disclaimer}

 \begin{center}
 \includegraphics[width=12.2cm]{images/panic.jpg}
 \end{center}


\end{frame}
 

\begin{frame}{EC-Schnorr and ECDSA}

SetUP() : Pick a \href{https://github.com/LedgerHQ/speculos/blob/master/src/bolos/cx_ec_domain.c}{{curve}} with parameters $(p,a,b,Gx,Gy,q)$ (\href{https://hyperelliptic.org/EFD/g1p/auto-shortw.html}{{ weierstrass equations and formulaes}} ).

  \begin{center}
\begin{tabular}{|c|c |c|}
\hline
Operation&\includegraphics[width=1cm]{images/schnorr.jpg} & ECDSA \\
\hline
KeyGen &$Q=xG$        &$Q=xG$ \\

Nonce$^*$&$k$	&  $k$ \\
Ephemeral&$R=kG$    &$R=kG$  \\
\hline

Hash &$e=H(m||R)$ & $e=H(m)$\\
\hline

Sign &$s=k-xe$    & $s=k^{-1}(e+xr)$  \\
     & $Sig=(R,s)$ & $Sig=(r,s)$ \\	
\hline
Verif &   $R'=sG+eQ$& $r'=(es^{-1}G+rs^{-1}Q)_x $ \\    
      & Accept if R'=R & Accept if r'=r \\
\hline
\end{tabular}  
 \end{center}
 
 
 \begin{center}
 \begin{small}
 {\emph{
 (* nonce generation may use \href{https://www.rfc-editor.org/rfc/rfc6979}{{RFC6979}} for misuse resistance)}}
 \end{small}
\end{center}

\end{frame} 
 

%%%%%%%%%%%%%%%%%%%%%%%%%%%%%%%%%%%%%%%%%%%%%%%%

\begin{frame}{Musig2: using Schnorr additive properties}

   
       

\only<1>
{  

Schnorr s part is linear in $(k,x)$ and { linerarity} is cool:
\begin{tabular}{lllr}
$s(k,x_1)+s(k,x_2)$&$=$& $s(k, x), $&$\forall  x=x_1+ x_2$\\
$s(k_1,x)+s(k_2,x)$&$=$& $s(k, x), $&$\forall k=k_1+ k_2$\\
\end{tabular}

(while ECDSA has degree two monomial in $(k,x)$)

 \begin{center}
\includegraphics[width=4cm]{images/multi3d.jpg}
\end{center}
    
}
     
\only<2>
{     
Linearity allow homomorphic additions. Idea: split X into $X=\sum a_iX_i$, k into $k=\sum k_i$.
 }

\only<3>
{

  \begin{center}
\begin{tabular}{|c|c |c|}
\hline
Operation&Schnorr & Insec\_Musig \\
\hline
KeyGen &$X=xG$       & $X_i=x_iG$ \\
{ KeyAgg} & - & X=$\sum_{i=0}^{n-1} a_iX_i$ \\
Nonce$^*$&$k$	&  $k_i$ \\
Ephemeral&$R=kG$   & $R_i=k_iG$ \\
{ Aggregate R}   & -     & $R=(\sum_{i=0}^{n-1} a_i.k_i).G=k.G$\\
Hash &$e=H(m||R)$ & $e=H(m||R)$\\
Sign &$s=k-xe$    & $s_i=k_i-a_ix_ie$  \\
{ Aggregate s} & - & $s=\sum s_i = k-xe$ \\
\hline
\end{tabular}  
 \end{center}
 
 }
 
\only<4>
{  
Musig2 uses a vectorial nonce of length $\mu$, injected in previous Insec\_Musig scheme.

  \begin{center}
\begin{tabular}{|c|c |c|}
\hline
Operation&Schnorr & Musig2 \\
\hline
KeyGen &$X=xG$       & $X_i=x_iG$ \\
{ KeyAgg} & - & X=$\sum_{i=0}^{n-1} a_iX_i$ \\
Nonce$^*$&$k$	&  $\vec{k_i}=(k_{i1}, \ldots , k_{i\mu})$ \\
Ephemeral&$R=kG$   & $\vec{R_i}=\vec{k_i}G$ \\
Hash Nonce & - & $b=H(X||R_0 \ldots R_\mu ||m)$\\

{ Aggregate R}   & -     & $R=\sum_{j=1}^\mu b^{j-1} (\sum_{i=0}^{n-1} a_i.k_i).G=k.G$\\
Hash &$e=H(m||R)$ & $e=H(m||R)$\\
Sign &$s=k-xe$    & $s_i=(\sum_{j=1}^\mu k_ijb^{j-1} )-a_ix_ie$  \\
{ Aggregate s} & - & $s=\sum s_i = k-xe$ \\
\hline
\end{tabular}  
 \end{center}
  
  
}
\end{frame}
  
%%%%%%%%%%%%%%%%%%%%%%%%%%%%%%%%%%%%%%%%%%%%%%%%

\begin{frame}{Musig2: Thresholdisation Principle}

\only<1>
{
Thresholdisation use the principle of \href{https://dl.acm.org/doi/10.1145/359168.359176}{ {Shamir's secret sharing scheme }}, which is in fact a reed solomon erasure code.

Goal: Given enough shares, it is possible to reconstruct the initial value.

\begin{center}
\includegraphics[width=4cm]{images/jump.jpg}
\end{center}            

}
\only<2>
{
Lagrange interpolation enables to switch from points to polynomial coefficients using the following formulaes:

\begin{center}
\begin{tabular}{cc}

\begin{minipage}{4cm}
\begin{center}
\includegraphics[width=4cm]{images/interpolation.jpg}
\end{center}            
\end{minipage}
&         
\begin{minipage}{4cm}
$$l_j(x)=\prod_{m\ne j}{x-x_m \over x_j-x_m}.$$
$$L(x)=\sum_{j=0}^k P(x_j)l_j(x).$$
\end{minipage}         
\\
\end{tabular}
\end{center}
The transformation L from $(P_0 \ldots P_k)$ to $(a_0 \ldots a_k)$ is a { linear} transformation in x.

\vskip+1cm
{\emph{Sidenote: This is closely related to the principle of FRI used in starks.}}

}
\only<3>
{
Key ideas:
\begin{itemize}
\item interprete aggregated secret key as a polynomial $P$ of degree $k$,
\item each share (user secret key) is a point of the polynomial,
\item blind the computation in the curve domain to perform the aggregation only handling public elements,
\item replace '$\sum_{i=0}^n$' in previous scheme by Lagrange polynomials,
\item some more steps are necessary (commitments) to avoid cheating.
\end{itemize}

Read \href{https://eprint.iacr.org/2020/852.pdf}{{ FROST}} for full description.

}



\end{frame}  


%%%%%%%%%%%%%%%%%%%%%%%%%%%%%%%%%%%%%%%%%%%%%%%%

\begin{frame}{Hacky Mul for secp256k1}


\begin{itemize}
\item  Point Multiplication is expensive in solidity (best implementation to date, FCL: 69K with huge precomputations, 200K otherwise). https://eprint.iacr.org/2023/939.pdf
\item  How to have a low gas Schnorr ?
\end{itemize}

\only<2>
{
Use a ZK verification: complex and expensive, but available in RAILGUN for the privacy property, not only computation.
}

\only<3>{
Original idea for k1 from the V: 

\includegraphics[width=8cm]{images/hacky.png}
}


\end{frame}  

\begin{frame}{Hacky Mul for secp256r1}

Inspired both by V and Y. El Housny: hinted mul for secp256r1 using 7951 verify;

We want to check that given $\alpha, Q$ the equality $alpha.G=Q$ holds.

ECDSA verification: 
$$(i,j)=(h.s^-1).G+(r.s^-1).Q$$
$$return ~~~i==r$$

To check that provided result (hint) $Q=\alpha.G$, set
$$(h,r,s,Q)= (1-\alpha)x, x, -x, Q)$$
Where $x$ is G x-ordinate.


if $ecdsa\_verify(h,r,s,q)=true$, then provided hint is correct.



\end{frame}  

%%%%%%%%%%%%%%%%%%%%%%%%%%%%%%%%%%%%%%%%%%%%%%%%

\begin{frame}{Conclusion}

\end{frame}  

%%%%%%%%%%%%%%%%%%%%%%%%%%%%%%%%%%%%%%%%%%%%%%%%

\end{document}
