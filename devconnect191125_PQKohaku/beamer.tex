
\documentclass[aspectratio=1610]{beamer}

\usepackage[utf8]{inputenc}
\usepackage[T1]{fontenc}
\usepackage{amsmath,amssymb,amsfonts}
\usepackage{hyperref} % before url package otherwise there is a pb
\usepackage{url}

\setbeamertemplate{navigation symbols}{%
    \usebeamerfont{footline}%
    \usebeamercolor[fg]{footline}%
    \hspace{1em}%
    \insertframenumber/\inserttotalframenumber
}
\setbeamercolor{footline}{fg=black}


\usepackage{comment}

% fix bug in algorithm2e
\makeatletter

\usepackage{tikz}
\usetikzlibrary{calc, positioning, arrows.meta}
\usetikzlibrary{shapes.geometric} % for ellipse
\usetikzlibrary{arrows}
\usetikzlibrary{positioning}
\tikzset{
  state/.style={
    rectangle,
    rounded corners,
    draw=black, thick,
    minimum height=2em,
    inner sep=2pt,
    text centered,
  },
}
\usetikzlibrary{tikzmark}
\usepackage{pgfplots}

\usepackage{multirow}
\usepackage{multicol}

\newcommand{\NN}{\ensuremath{\mathbb N}}
\newcommand{\ZZ}{\ensuremath{\mathbb Z}}
\newcommand{\QQ}{\ensuremath{\mathbb{Q}}}
\newcommand{\RR}{\ensuremath{\mathbb{R}}}
\newcommand{\CC}{\ensuremath{\mathbb{C}}}
\newcommand{\K}{\ensuremath{\mathbb{K}}}
\newcommand{\PP}{\ensuremath{\mathbb{P}}} % for projective line
\newcommand{\FF}{\ensuremath{\mathbb F}}
\newcommand{\F}{\ensuremath{\mathbb F}}
\newcommand{\GG}{\ensuremath{\mathbb G}}
\newcommand{\G}{\ensuremath{\mathbb G}}
\newcommand{\GT}{\ensuremath{\mathbb G_{\text{T}}}}
\newcommand{\Fp}{\F_p}
\newcommand{\cO}{\ensuremath{\mathcal O}}
\newcommand{\cF}{\ensuremath{\mathcal F}}
\DeclareMathOperator{\Norm}{Norm}
\DeclareMathOperator{\ord}{ord}
\DeclareMathOperator{\disc}{Disc}
\DeclareMathOperator{\coeff}{Coeff}
\DeclareMathOperator{\cont}{cont}
\DeclareMathOperator{\reslt}{Res}
\DeclareMathOperator{\Reslt}{Res}
\DeclareMathOperator{\Res}{Res}
\DeclareMathOperator{\val}{val}
\DeclareMathOperator{\GF}{GF}
\DeclareMathOperator{\aut}{aut}
\DeclareMathOperator{\End}{End}
\DeclareMathOperator{\lc}{lc} % for leading coefficient
\DeclareMathOperator{\Id}{Id} % for Identity
\DeclareMathOperator{\Div}{div}
\newcommand{\BLS}{\mbox{BLS}}

\usepackage{fontawesome}

\date{November 18th, 2025 -- Buenos Aires, DevConnect}
\title{
  Kohaku Wallet: Post-Quantum Smart Accounts on Ethereum Today}
\author{\textbf{Simon Masson}, Renaud Dubois\\
ZKNox\\
\includegraphics[height=4em]{zknox.png}\\
Privacy and Compliance Summit}
\begin{document}
\begin{frame}
  \maketitle
\end{frame}

\begin{frame}{ZKNOX Casting}
  \begin{minipage}{.65\linewidth}
   \begin{tabular}{rl}
      \multirow{4}{*}{\includegraphics[width=2cm]{NB.jpeg}} & \textbf{Nicolas Bacca, "Chief"}\\
                               & \small{$20^+$ years experience ($10^+$y web3)}\\
                               & Security and hardware specialist\\
                               & \small{Prev.~Ledger cofounder/CTO}\\
                               & \\
      \multirow{4}{*}{\includegraphics[width=2cm]{RD.jpeg}} & \textbf{Renaud Dubois, "Agent Smith"}\\
                               & \small{$20^+$ years experience ($3^+$y web3)}\\
                               & Cryptographer\\
                               & \small{Prev.~Ledger, Thales}\\
                               & \\
      \multirow{4}{*}{\includegraphics[width=2cm]{SM.jpg}} & \textbf{Simon Masson, "Bizut"}\\
                               & \small{$8^+$ years experience ($4^+$y web3)}\\
                               & Cryptographer\\
                               & \small{Prev.~Heliax, Thales}\\
  \end{tabular}
\end{minipage}\pause
\begin{minipage}{.33\linewidth}
   Expertise and innovation to every challenge on the whole security chain:
   \begin{itemize}
   \item user end\\ (secure enclaves, hardware wallets),
   \item back end\\ (TEE, HSMs),
   \item on-chain\\ (smart contracts).
   \end{itemize}

  \pause
 
  \vspace{1em}

  \href{https://zknox.eth.limo/}{\texttt{https://zknox.eth.limo/}}

  \vspace{1em}

  \href{https://github.com/zknoxhq/}{\texttt{https://github.com/zknoxhq/}}

\end{minipage}
\end{frame}


\begin{frame}{Summary}
 \tableofcontents
\end{frame}  

% \begin{frame}{Content of this talk}
%   \begin{enumerate}
%     \item Introduction: privacy in Ethereum
%     \item Zero-knowledge circuits in a nutshell
%     \item Elliptic curve in circuits
%     \item Hinted scalar multiplications
%     \item Conclusion and perspectives
%   \end{enumerate}
% \end{frame}
 \section{ction: Quantum Apocalypse }
    %\frame{\sectionpage}
    
%%%%%%%%%%%%%%%%%%%%%%%%%%%%%%%%%%%%%%%%%%%%%%%%    
\begin{frame}{Introduction: Quantum Apocalypse}

Shor algorithm solves factorization and discrete logarithm problems. All current authentication systems are cooked as soon as Quantum computer rises.

\bigskip
\begin{center}
\includegraphics[width=6cm]{images/Apocalypse.png}
\end{center}
\end{frame}


%%%%%%%%%%%%%%%%%%%%%%%%%%%%%%%%%%%%%%%%%%%%%%%%    
\begin{frame}{When ?}


\only<1>{
\begin{center}
\includegraphics[width=12cm]{images/when1.png}
\end{center}
}

\only<2>{
\begin{center}
\includegraphics[width=12cm]{images/when2.png}
\end{center}
}

\only<3>{
\begin{center}

Federal agencies:  

NIST: 2035
ANSSI/BSI: 2030

Blockchains might not care about feds, but for stablecoins regulation, those might be hard deadlines.

\end{center}
}

\end{frame}

%%%%%%%%%%%%%%%%%%%%%%%%%%%%%%%%%%%%%%%%%%%%%%%%    
\begin{frame}{EF got your back}

\begin{center}
\includegraphics[width=6cm]{images/lean.jpg}
\end{center}

\end{frame}
%%%%%%%%%%%%%%%%%%%%%%%%%%%%%%%%%%%%%%%%%%%%%%%%    
\begin{frame}{Candidates}

\begin{tabular}{ccccc}
\hline
Complexity metric & FALCON512 & DILITHIUM2 & SPHINCS+ & GemSS \\
Public Key size & 897 & 1312 & 32 & 33 \\
Private Key size & 7553 & 2528& 32 & 64 \\
Signature size & 666 & 2420 & 17088 & 21952 \\
\hline
\end{tabular}
  
\bigskip
Current proposals:
\begin{itemize}
\item FALCON: EIP8052/EIP7619
\item Dilithium: EIP8051
\end{itemize}

Difference for 8052/7619: zk friendly hash (separate core from hashing to domain), 7932 specification

\bigskip

8051 and 8052 comes with contracts, signers, and hardware signer (8051 only).
(Wait for next talk for the onchain demonstration).

\end{frame}


%%%%%%%%%%%%%%%%%%%%%%%%%%%%%%%%%%%%%%%%%%%%%%%%    
\begin{frame}{Advertisement}


All is delivered as public good, experiment, integrate, give feedback, hack this week end (ETHGLOBAL).


\begin{center}
\includegraphics[width=14cm]{images/qr.png}
\end{center}

A PQ-vault, staking ETH (gas cost is high, better suited for high amount with few movements)

\end{frame}
%%%%%%%%%%%%%%%%%%%%%%%%%%%%%%%%%%%%%%%%%%%%%%%%    
\begin{frame}{Remarks}

\begin{itemize}  
\item Authentication might be solved a bit later, (but we shall on the shelf
solution).

\item \textcolor{red}{Confidentiality shall be solved NOW.}

\end{itemize}

\bigskip

Ethereum components at risk:
\begin{enumerate}  

\item EoA private keys (notably using ECDSA)
\item \textcolor{red}{Private Payments (Privacy Pool, TC, sRAILGUN)}
\item BLS signatures in consensus (lean CL)
\item Data Availability Sampling (leveraging KZG commitments)
\end{enumerate}

ZKNOX provides solution for first point, and experimentations results for 2 and 3.

\end{frame}
%%%%%%%%%%%%%%%%%%%%%%%%%%%%%%%%%%%%%%%%%%%%%%%%    
\section{Verifiers SC implementation}

\begin{frame}{Progressive Roadmap}

  Verifiers:
\begin{itemize}
\item \textcolor{green}{Step1: use Account Abstraction (EIP-7702+7579/4337) with full
solidity to experiment}
\item \textcolor{orange}{Step2: benchmark in nodes  (validate gas hypothesis)}

\item {Step3: EIP accepted }

\item remove eoA (EIP-7701/EIP-7560)


\end{itemize}  

Signers:
\begin{itemize}
\item \textcolor{green}{Step1: Software signers}
\item \textcolor{orange}{Step2: hardware signers }

\item remove eoA (EIP-7701/EIP-7560)


\end{itemize}  


\end{frame}


\begin{frame}{Verifiers: implementation results}

\begin{tabular}{ccc}

Function & Description & Gas Cost \\
\hline

TETRATION ethfalcon.verify & EVM Friendly* & 24M \\
\hline


ZKNOX falcon.verify & NIST & 7M \\
\hline

ZKNOX ethfalcon.verify & EVM Friendly & 1.8 M \\
\hline

ZKNOX epervier.verify & Recover EVM friendly &  1.9M \\
\hline
\end{tabular}




\begin{tabular}{ccc}

Function & Description & Gas Cost \\
\hline


ZKNOX dilithium.verify & NIST & 13.3M \\
\hline

ZKNOX ethfalcon.verify & EVM Friendly & 6 M \\
\hline

\end{tabular}




\end{frame}  

%%%%%%%%%%%%%%%%%%%%%%%%%%%%%%%%%%%%%%%%%%%%%%%%    
\section{Signers implementation}


\begin{frame}{Signers implementation}

\begin{tabular}{cccc}
  \hline
Scheme & Operation & RAM Consumption & Code size \\
FALCON & Keygen & 40 Kb & 30-50 Kb \\
       & Signing & 40 Kb & 30-50 \\
\hline
Dilithium & Keygen & 7 Kb & 30-50 Kb \\
       & Signing & 7 Kb & 30-50 \\
       \hline
\end{tabular}
\bigskip

Low Ram implementation of Dilithium is possible, falcon reach the limit for a SE implementation.
\bigskip

Next experiment: solve RAM limitation with ORAM (extend RAM with external ciphering on host).
Part of another ZKNOX implication in Kohaku: universal Ethereum Application.


\end{frame}  
%%%%%%%%%%%%%%%%%%%%%%%%%%%%%%%%%%%%%%%%%%%%%%%%    

\section{Validity and Privacy lattices results}

\begin{frame}{Proving FALCON and DILITHIUM}

Need
\begin{itemize}
\item  Allow FALCON/DILITHIUM integration into ZKEVM
\item  Allow PQ-Private Payments (PQ-Railgun)
\item Solution for BLS replacement is Batching with snarks/starks.
\end{itemize}

Second one could use a dedicated scheme (LABRADOR, Lean CL team work).


\begin{itemize}
\item FALCON and DILITHIUM operates on non native fields (Babybear,
STwo)
\item  Non Native Fields require x30 more constraints (source:
Bandersnatch vs P256 in Gnark, https://eprint.iacr.org/2025/933)
\item Solution for BLS replacement is Batching with snarks/starks.
\end{itemize}

\end{frame}  


%%%%%%%%%%%%%%%%%%%%%%%%%%%%%%%%%%%%%%%%%%%%%%%%    
\begin{frame}{ZK friendly FALCON/DILITHIUM}

Obvious tweaks

\begin{itemize}
  \item Replace Hash function by a ZK friendly one
 \item Provide hints (see epervier definition) for easy batch inversion
\item Accumulate several NTT steps, reduce once (Credit to Zhenfei)
\end{itemize}

Harder tweaks

\begin{itemize}
\item  Starks fields are going shorter (no accumulation trick)
Replacing fields is less trivial than for ECC (pick prime order curve
and twist)
\item FALCON is prone to overstretch attacks for ZK fields
\item Estimator for S2/Babybear on dilithium provides 20% slowdown
factor (WIP)
\end{itemize}

\end{frame}  

%%%%%%%%%%%%%%%%%%%%%%%%%%%%%%%%%%%%%%%%%%%%%%%%    

\begin{frame}{Results}

  \begin{tabular}{cccc }
    \hline
  Function	&Description	&gas cost	& Max per block \\
  \hline
  ECDSA & Starkware built in & - & 3000 \\
  \hline
ETH\_FALCON	&ZKnox Use Keccak-ctr	&102M	&10 \\
\hline
ETH\_FALCON	&Starkware+ ZKNOX Keccak-ctr	&340M	3 \\
\hline
FALZKON	Use &Blake2s-ctr	&41M	&25 \\
\hline
  \end{tabular}

Result for a high level Cairo, dedicated AIR shall dramatically reduce this factor.

\bigskip
Dilithium requires less hashing and more easily tuned : better zk candidate.

\bigskip


https://github.com/ZKNoxHQ/falzkon


\end{frame}

%%%%%%%%%%%%%%%%%%%%%%%%%%%%%%%%%%%%%%%%%%%%%%%%    
\section{Conclusion}

\begin{frame}{Accomplishments}

\begin{itemize}
\item dramatical reduction of complexities (ZK and SC verifiers)
\item first full suite with secure element PQ signer to on chain contract verification (8051)
\item 

\end{itemize}  

\end{frame}  
%%%%%%%%%%%%%%%%%%%%%%%%%%%%%%%%%%%%%%%%%%%%%%%%    
\begin{frame}{Conclusion}

\textcolor{green}{Easy}
  \begin{itemize}
\item Easy solutions for EOA migration (use AA+ 8051/8052)
\item 8051 and 8052 proposals, help us with ACD 
\item PQ ciphering (view keys) is easy (just more storage cost)
\end{itemize}

FALCON is greater for plain constraints, less for ZK/MPC/signers constraints.

\bigskip

\textcolor{red}{Hard}
\begin{itemize}
\item no succinct replacement of pairings (consensus + private Payments systems)
\item almost probably no constant size PQ replacement
\end{itemize}

Addiction to 0(1) bandwidth will strongly strike those part of the protocol. 

\bigskip

Now let's witness the first full suite with a HW signer and on chain PQC transaction.


\end{frame}

%%%%%%%%%%%%%%%%%%%%%%%%%%%%%%%%%%%%%%%%%%%%%%%%    
\end{document}

